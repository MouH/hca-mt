\documentclass[utf8,english]{gradu3}
% If you are writing a Bachelor's Thesis, use the following instead:
%\documentclass[utf8,bachelor,english]{gradu3}

\usepackage{booktabs} % good for beautiful tables
\usepackage{todonotes}
\usepackage{url}
\usepackage{algorithm}
\usepackage{algpseudocode}
\usepackage{amssymb}
%some hacking because both the template and the amsmath package define \proof
\let\proof\relax 
\let\endproof\relax
\usepackage{amsmath}
\usepackage{amsthm}
\usepackage{graphicx}
\usepackage{paralist}
\usepackage{mdwlist}
\usepackage{threeparttable}
\usepackage{footmisc}
\usepackage{soul}
\usepackage{cleveref}
\usepackage{varwidth}
\usepackage{enumitem}



% NOTE: This must be the last \usepackage in the whole document!
\usepackage[bookmarksopen,bookmarksnumbered,linktocpage]{hyperref}

\addbibresource{malliopas.bib} % The file name of your bibliography database


\begin{document}

\title{Clustering Analysis and Approximate Hierarchy Clustering Algorithm}
\translatedtitle{\LaTeX-tutkielmapohjan {gradu3} käyttö}
\studyline{All study lines}
\avainsanat{%
  \LaTeX,
  {gradu3},
  pro gradu -tutkielmat,
  kandidaatintutkielmat,
  käyttöohje}
\keywords{\LaTeX, {gradu3}, Master's Theses, Bachelor's Theses, user's guide}
\tiivistelma{%
  Empty TODO
}
\abstract{%
  Clustering Analysis and Approximate Hierarchy Clustering Algorithm.
}

\author{Mou Hao}
\contactinformation{Ag~C416.1, 
	\texttt{mouhao1990@outlook.com}}
% use a separate \author command for each author, if there is more than one
\supervisor{Unsupervised work}
% use a separate \supervisor command for each supervisor, if there
% is more than one

 % you don't need this line in a thesis
% \type{Template and manual for a thesis document class}

\maketitle

\preface
This is where you can write a preface for your thesis.  Most theses
don't have prefaces, but if you write one, keep it short (at least one
page).

The preface should discuss more the thesis process than the content of
the thesis.  For example, if there is something out of the ordinary in
your choice of a thesis topic or if something out of the ordinary
happened during its prepararion, the preface is where you could write
about it.  It is also customary in a preface to thank by name those
persons who helped you with your thesis -- at least your supervisor,
your spouse and your children, if any.  (Your family likely will have
helped you by encouraging and supporting you.)

The preface is typically in the first person (``I'').  It is also common
to sign it.

Jyväskylä, \today

\bigskip

The Author

\begin{thetermlist}
\item[\TeX] A batch-oriented typesetting system written by 
Donald Knuth in 1977--1989 \parencite[see][]{knuth86:_texbook}. 
\item[\LaTeX] A system, built on top of \TeX\
  \parencite{knuth86:_texbook}, for typesetting structured
  documents \parencite[see][]{lamport94:_latex}.  Its current version
  is \LaTeXe.
\end{thetermlist}

\mainmatter

\chapter{Introduction}

Introduction (describe the structure of the thesis and point out the contributions)

\chapter{Clustering}

Clustering (includes normal and hierarchical clustering)

The clustering methods are mainly divided into two categories viz., partition based clustering and hierarchical clustering, based on the way they produce the results.

\section{Similarity Measure}

Similarity measure, which is the essential part of any clustering algorithm, a brief but concrete discussion about similarity measure is conducted in this section.

\section{Partial Clustering}

Partial clustering:
- distance based 
	k-means
- density based

\section{Hierarchical Clustering Algorithm}

discuss about the basic idea of hierarchical clustering algorithm.
discuss the usage of HCA
show the basic single  average linkage algorithm by naive code.

\algrenewcommand\Return{\State \algorithmicreturn{} }% http://tex.stackexchange.com/questions/69449/avoid-putting-statements-on-the-same-line-with-algorithmicx
\begin{algorithm}[h]
	\caption{Primitive AHC algorithm}
	\label{alg:primitiveHAC}
	\begin{algorithmic}[1]
		\Procedure{primitive AHC}{s,d}
		
		\State $S_{origin} \leftarrow S$
		\State $n \leftarrow \lvert S \rvert$
		\State $den \leftarrow \left[\right]$
		\State $size[x] \leftarrow 1, $ for all $x \in S$
		\For{$i \leftarrow 0, \dots , n - 2 $}
		\State $(I,J) = argmin_{(SxS)\setminus \Delta} d$
		\State append (I,J) to den
		\State $S \leftarrow S \setminus \left\{I,J\right\} $
		\State Create a new label $L$, $L \notin S \cup S_{origin}$
		\State \begin{varwidth}[t]{\linewidth}
			Update the matrix containing the distances
			\par\hskip\algorithmicindent $d[L,x] = d[x,L] = \text{FORMULA} ( d[I,x], d[b,J],$ 
			\par\hskip\algorithmicindent $d[I,J], size[I], size[J]) $,  for all $x \in S$
		\end{varwidth}
		\State $size[L] \leftarrow size[I] + size[J]$
		\State $S \leftarrow S \cup \left\{L\right\}$
		\EndFor 
		
		\Return den
		\EndProcedure
		
		The $\text{FORMULA}$ is the updating formula used for the chosen linkage, while $d$ is the distance metric. Note that our notation somewhat freely uses $I$ and $J$ to mean either the label of the cluster or the cluster itself.
	\end{algorithmic}
\end{algorithm}

\chapter{Approximate Hierarchical Clustering Algorithm}

As discussed in the previous chapter, the time complexity of a traditional accurate hierarchical clustering algorithm is $O(n^3)$. However, the datasets size of today's application domain has dramatically increased. Conducting a traditional agglomerative hierarchical clustering on such dataset would not be proper for an expected running time.

To handle the scaling challenge, researchers found two approaches. At fist, researchers are focused on how to find faster algorithms which generate the same hierarchical tree as the original algorithm. The hardworking of the first approach came to its limit, when dataset scale becomes even lager. The later contribution turns to find a approximate hierarchical clustering algorithm, which is not consistency to but closely resembles the exact algorithms. 

This chapter will mainly discuss the current research in the approximate hierarchical clustering algorithms. 

\section{Experiment and Comparison Difficulty}

Before the discussion of the approximate algorithms, an brief illustration about the difficulty of experiment and comparison of such approximate hierarchical clustering algorithms.

\section{Algorithm by Patra}

Patra et al.~\cite{patra2010distance}  proposed a method, l-AL for AHC to deal with large dadtasets problem with average linkage. In their work, a set of leaders are proposed to represent the whole datasets, which are then applied the standard average link method. The advantage is that this method works for any distance metric, and reduces the running complexity as it is not requested to store the whole  dataset in to memory, only the leaders are retained. 

To perform a l-AL algorithm, firstly, a set of leaders should be chose, and a standard average linkage method will be implement inside each group. The average linkage method has been discussed in the previous chapter. The focus below will be mainly about the way they used to choose leaders from the origin dataset.

\algrenewcommand\Return{\State \algorithmicreturn{} }
\begin{algorithm}[h]
	\caption{Leaders Selecting algorithm}
	\label{alg:selectLeader}
	\begin{algorithmic}[1]
		\Procedure{leaderSelect}{dataset, $\tau$}
		\State $leaders \leftarrow empty hash map of node and set of nodes$
		\For{$i \leftarrow dataset$}
			\State \text{IF}  exists a $l$ in $leaders$, and $ || l - i || < \tau $
			\State \hskip\algorithmicindent put current node $i$ into the set in $leaders$ with key $i$.
			\State \text{ELSE}
			\State \hskip\algorithmicindent make i a new leader, put an empty set into $leaders$ with key $i$
		\EndFor		
		\Return $leaders$
		\EndProcedure

	This algorithm takes two input, one is the whole dataset which is used in the hierarchical clustering, the other one is a tolerante value $\tau$. The return value is the chosen leaders with their attached nodes.
	\end{algorithmic}
\end{algorithm}


\section{Algorithm by Gilpin}
3.2 Algorithm by Gilpin et. al

\section{Twister Tries Approach}
3.3 Twister tries + the extension I worked on later

\section{Comparison}
3.4 Some comparison (this can also be done directly in the section, but it might be easier to do it separatly)

Appendix: the article

I will here assume that you know the basics of using the \LaTeX\
system.  The original \LaTeX\ book \parencite{lamport94:_latex} is the
official manual.  There are also a lot of books in English about using
\LaTeX.  I have also written one in
Finnish \parencite{kaijanaho03:_latex_ams_latex}.\footnote{Many \TeX\
  and \LaTeX\ books use a cat figure in their cover.  The cover
  picture of my own book was rather abstract; see
  Figure~\ref{fig:opus-kissa}.}  A good English guide, freely
available on the Internet, is \textit{The Not So Short Introduction to
  \LaTeXe} \parencite{oetiker:_not_so_short_introd_latex}.  Remember
to read the \LaTeX\ source ode of this sample, not just the typeset
version (eg.~PDF).

Please note that the instructions given in this sample are by no means
official.  Always follow your supervisor's instructions even if they
conflict with what this sample says.


\chapter{The structure of the thesis}

There should be 5--9 numbered chapters in a thesis, including
Introduction and Conclusion.  If necessary, you can use sections
and subsections to give the thesis a more fine-grained structure.

The chapters that lie between Introduction and Conclusion are
sometimes collectively called the \textit{body} of the thesis.  It is
often said to start with a \textit{theoretical part}, which is then
followed either a \textit{main theorem}, a \textit{constructive part}
or a \textit{empirical part}.


\section{The theoretical part}

The goal of the theoretical part of a thesis is to develop the
theoretical background required in the thesis.  The idea is that a
reader of the thesis should, based on just the thesis itself, be able
to understand all the special concepts and methods used in the thesis.
A good thesis also gives well-argued reasons for why exactly these
concepts and methods are in use in the thesis (with the main
alternatives given in the literature mentioned).

The best way to present and use the theoretical bakcground depends on
what the thesis is like.  The theoretical part of a
mathematico-theoretical work differs considerably fron the theoretical
part of a constructive software development work; quite different from
both is the theoretical part of a quantitative or qualitative
empirical study that is based on the traditions of the behavioral or
the social sciences.  Reading other theses of the same type, as well
as similar published research reports, will give you a good impression
of what is required of your own thesis.

\section{After the theory}

The theoretical part is followed by your contribution:
\begin{itemize}
\item In a mathematico-theoretical thesis it is usually a sequence of
  definitions and lemmas of your own devising, which then culminate in
  the proof of your main theorem.
\item In a constructive thesis it is usually a computer program or
  other artefact that you have made yourself.
\item In an empirical thesis it is a set of empirical results obtained
  by applying a empirical research method.
\end{itemize}

You should present your contribution with precision, giving reasons
for the choices you have made.  You should follow the best practices
of the research tradition you are using.

\chapter{Using the literature}

The theoretical part is almost always based solely on the literature.
When discussing your contribution, you may also need to cite the
literature.

Remember to avoid plagiarism.  If you copy, either verbatim or with
slight changes (or, example, in your own translation) text from some
source, make it clear to the reader.  Mark your quotes (using
quotation marks or some other clear manner) and give a precise
citation.  If you do not quote verbatim, mark any changes you have
made.  In most situations, however, it is better to use your own
words, based on more than one source.  Even then, give clear
citations.

The {gradu3} document class automatically uses the \textsc{Bib\LaTeX}
system \parencite{biblatex-manual} and it Chicago
style \parencite{biblatex-chicago-manual}.  You can switch off this
automation by using the \string\documentclass-option manualbib, but
that means you have to take care of the bibliography yourself, and the
techniques discussed here may not be available.  Please note that the
Department recommends using a Chicago style for your bibliography.

\section{Citations}

You can cite sources in two ways.  First, you can use the citation as
a noun: \textcite[Chapter~8.8.4]{aho-compilers} briefly discuss the
use of graph coloring in the register allocation phase of a compiler.
In this case, use the \string\textcite\ command.  Second, you can use
a citation as a parenthetical, which is not read aloud: Graph coloring
is one possibile way to allocate
registers \parencite[Chapter~8.8.4]{aho-compilers}.  Use the
\string\parencite\ command for this.

Both commands (\string\textcite\ and \string\parencite) take three
parameters, two of which are optional.  The first (optional) parameter
is a pre-note, the second (optional) parameter is a post-note, and the
third (mandatory) parameter is the citation
key \parencite[see][Section~3.7]{biblatex-manual}.  The citation in
the preceding sentence was made using the following command:

\begingroup\footnotesize
\begin{verbatim}
\parencite[see][Section~3.7]{biblatex-manual}
\end{verbatim}
\endgroup

If you give these commands just one optional argument (that is, one
enclosed in square brackets), it will be interpreted as a post-note.
If you want to give only a pre-note, leave the post-note empty
\parencite[see][]{biblatex-manual}:

\begingroup\footnotesize
\begin{verbatim}
\parencite[see][]{biblatex-manual}
\end{verbatim}
\endgroup

It is also possible to cite multiple sources in the same citation
%
\parencites%
  [see][Section~3.7]{biblatex-manual}%
  [regarding citations in general, see also][Section~5.3.2]%
    {biblatex-chicago-manual}%
\relax.
%
Use the command  \string\parencites\
for this.  For each citation, give it the same parameters as you would give
a single \string\parencite\
command.  It is good practice (but often not necessary) to end the command
in a \string\relax, so that no surprises ensue.

\begingroup\footnotesize
\begin{verbatim}
\parencites%
  [see][Section~3.7]{biblatex-manual}%
  [regarding citations in general, see also][Section~5.3.2]%
    {biblatex-chicago-manual}%
\relax.
\end{verbatim}
\endgroup

If you break the command into multiple lines, use the comment sign
to end each line, to prevent spurious spaces.

\section{The bibliography database}

You should add all the sources you want to cite in a separate
bibliography database written on the \textsc{Bib\TeX} format.  You can
use many bibliographical tools in creating and maintaining it, but it
is perfectly possible to write it by hand.  The name of your
bibliography database must be given as an argument to the
\string\addbibresource\ command.

The database in \textsc{Bib\TeX} format is a text file following
special formatting rules.  It consists of records, each of which
starts with an @~sign, which is then followed by the type of the
record.  The rest of the record goes inside curly braces.  For example,
the compilers book cited earlier \parencite{aho-compilers} can be
represented as the following record:

\begingroup\footnotesize
\begin{verbatim}
@Book{aho-compilers,
  author =       {Alfred V. Aho and Monica S. Lam and Ravi Sethi and
                  Jeffrey D. Ullman},
  title =        {Compilers},
  subtitle =     {Principles, Techniques, \& Tools},
  publisher =    {Pearson Addison Wesley},
  year =         2007,
  address =      {Boston},
  edition =      2
}
\end{verbatim}
\endgroup%

The type of this record is ``book''.  The first word inside the curly
braces is the citation key, which is used in the \string\textcite\ and
\string\parencite\ commands.  It is followed by a comma and a set of
named fields like ``author'', ``title'', ``subtitle'' and
``publisher''.  The content of the field is written inside curly
braces, although numerical data can be written without them.

The names of the authors are written mainly in the conventional way.
An alternative is to invert it, giving the surname first, followed by
a comma and the first name (``Aho, Alfred V.''), and in some special
cases this is mandatory.\footnote{For example, if the author has a
  double surname without a hyphen separating them; as one example, the
  name of Simon Peyton Jones should be written in the database as
  ``Peyton Jones, Simon''.}  If there are multiple authors, their
names must be separated by an ``and''.  If you do not list all
authors, put ``and others'' after the last listed name.

If the author of some source is an organization, its name must be written
within another set of curly braces \parencite[eg.][]{unicode620}:

\begingroup\footnotesize
\begin{verbatim}
@Book{unicode620,
  author =       {{Unicode Consortium}},
  title =        {The Unicode Standard, Version 6.2.0},
  year =         {2012},
  url =          {http://www.unicode.org/versions/Unicode6.2.0/},
  urldate =      {2013-01-29}
}
\end{verbatim}
\endgroup

If a source, for some reson, has no named author, leave the ``author''
field out ntirely.  In that case, the citation uses the source's
title \parencite[eg.][]{presidential-novel}:

\begingroup\footnotesize
\begin{verbatim}
@Book{presidential-novel,
  title =        {O},
  subtitle =     {A Presidential Novel},
  publisher =    {Simon \& Schuster},
  year =         {2011},
}
\end{verbatim}
\endgroup

A journal article \parencite[eg.][]{strachey-fundamentals} is given a
record like the following:

\begingroup\footnotesize
\begin{verbatim}
@Article{strachey-fundamentals,
  author =       {Christopher Strachey},
  title =        {Fundamental Concepts in Programming Languages},
  journal =      {Higher-Order and Symbolic Computation},
  year =         2000,
  volume =       13,
  number =       {1--2},
  pages =        {11--49},
  doi =          {10.1023/A:1010000313106}
}
\end{verbatim}
\endgroup

Note especially the field ``doi'', in which you can write the Digital
Object Idenifier (DOI) of the article.  It is usually a better choice
than any URL, as the DOI is a permanent identifier for the article.
Most DOIs are also convertible to URLs by prepending
\url{http://dx.doi.org/}.

If the DOI of an online source is not known (or there is none at all),
you can use the ``url'' field.  In that case, you should also give the
date on which you read the source, in the field ``urldate'' (using the
international standard format YYYY--MM--DD).  You should choose the
address with great care, so that it is as precise as possible and
remains valid as long as possible.  If the page has a specially
indicated permanent link (or permalink), use it.

When citing a WWW page that is not a book or an article or any other
formal publication, you can use the ``online'' record
type \parencite[eg.][]{debian-social-contract}:

\begingroup\footnotesize
\begin{verbatim}
@Online{debian-social-contract,
  title =        {Debian Social Contract},
  year =         {2004},
  url =          {http://www.debian.org/social_contract.en.html},
  urldate =      {2013-01-29}
}
\end{verbatim}
\endgroup

Some sources are edited collections of independent articles.  In that
case, you should generally cite a specific article in
it \parencite[eg.][]{prechelt-credibility} instead of the full
collection.  Even then, you should add both the collection and the
cited article as their own records, and use a ``crossref'' field in
the article record to refer to the collection:\footnote{It is
  permissible to combine the article and the collection into one
  InCollection record, for example if one cites only one article in
  the collection.  In that case, the title of the collection goes in a
  ``booktitle'' field, and no ``crossref'' field is used.}

\begingroup\footnotesize
\begin{verbatim}
@Collection{making-software,
  editor =       {Andy Oram and Greg Wilson},
  title =        {Making Software},
  subtitle =     {What Really Works, and Why We Believe It},
  publisher =    {O'Reilly},
  year =         2011
}
@InCollection{prechelt-credibility,
  author =       {Lutz Prechelt and Marian Petre},
  title =        {Credibility, or Why Should I Insist on Being
                  Convinced},
  crossref =     {making-software},
  pages =        {17--34}
}
\end{verbatim}
\endgroup

Note that a collection has an ``editor'' instead of an ``author''.

For more information about the structure of a bibliography databasem
see the \textsc{Bib\TeX} manual \parencite{bibtexing},
the \textsc{Bib\LaTeX} manual \parencite[Section~2]{biblatex-manual},
and the \textsc{Bib\LaTeX}-Chicago manual
\parencite[Sections 5.1--5.2]{biblatex-chicago-manual}.  There are
also more examples in the source code of this document.


\section{The bibliography}

The bibliography database is converted into the bibliography by using
the utility program {biber}.  It is fairly new, and is often missing
from machines whose \TeX\ installation is not up to date.  Of the
ssh-accessible Linux servers of the University, only charra.it.jyu.fi
has it at this time.  It is installable in Ubuntu since version~12.10
(Quantal Quetzal) and in Debian since version~7 (Wheezy).  For
Windows, use the 32-bit Mik\TeX\ package
miktex-biber-bin.\footnote{Last I looked, there was no 64-bit package
  of biber for Mik\TeX.}

On the command line, biber is simple to use.  Once \LaTeX (or
pdf\LaTeX) has been run once, invoke biber with the document name
(without the .tex part) as its argument.  After that, run \LaTeX\ (or
pdf\LaTeX) at least once, until the latest run does not request
another run.  For example:

\begingroup\footnotesize
\begin{verbatim}
$ pdflatex malliopas
[...]
Package biblatex Warning: Please (re)run Biber on the file:
(biblatex)                malliopas
(biblatex)                and rerun LaTeX afterwards.
[..]
Output written on malliopas.pdf (18 pages, 96855 bytes).
Transcript written on malliopas.log.
$ biber malliopas
INFO - This is Biber 0.9.9
[...]
INFO - Output to malliopas.bbl
$ pdflatex malliopas
[...]
LaTeX Warning: Label(s) may have changed. Rerun to get cross-references right.
[...]
Output written on malliopas.pdf (21 pages, 107373 bytes).
Transcript written on malliopas.log.
$ pdflatex malliopas
[...]
Output written on malliopas.pdf (21 pages, 107509 bytes).
Transcript written on malliopas.log.
\end{verbatim}
\endgroup

\section{Known problems}

The \textsc{Bib\LaTeX} version 2.6 (released April~30, 2013) has a bug
causing the following error message:%
{\footnotesize%
\begin{verbatim}
Runaway argument?
{bibliography = {{Kirjallisuusluettelo}{Kirjallisuus}}, references = \ETC.
! Paragraph ended before \DeclareBibliographyStrings was complete.
\end{verbatim}
}%
This bug was fixed in the following version, 2.7 (released July~7,
2013).  If upgrading is not an option, there is a simple fix.  Look in
the file \texttt{.../biblatex/lbx/finnish.lbx} for the line
{\footnotesize%
\begin{verbatim}
editorsan        = {{toimittaneet ja selityksin varustaneet,% FIXME: unsure
\end{verbatim}
}%
Edit the line to look like this:
{\footnotesize%
\begin{verbatim}
editorsan        = {{toimittaneet ja selityksin varustaneet}% FIXME: unsure
\end{verbatim}
}%
(Replace the comma with a closing curly brace.)


\chapter{Special properties of the document class}

Generally, {gradu3} behaves like the report document class that is shipped
with \LaTeX.  There are, however, some differences:
\begin{itemize}
\item You do not need to load the packages {inputenc}, {fontenc},
  and {babel}.
  \begin{itemize}
  \item You must indicate the character set you are using by giving it as
    an option to the
    {\string\documentclass} command.  Nowadays {utf8} 
    is generally a good choice, although some situations may require
    using latin1 or latin9.
  \item If your thesis is written in English, indicate this using the
    option english to the {\string\documentclass} command.  (The
    default is Finnish.)
  \end{itemize}
\item If you are writing a Bachelor's Thesis, use the option bachelor to the
  {\string\documentclass} command.
\item Specify the metadata of your thesis using the commands given in
  Table~\ref{tbl:metatiedot}.  They must be given before the
  {\string\maketitle} command.
\begin{table}[h]\centering
  \begin{tabular}{lp{9cm}}
    \toprule
    Command & Meaning \\
    \midrule
    {\string\title}
    & The title of the thesis (do not use the {\string\thanks} command) \\
    {\string\translatedtitle}
    & The Finnish title of an English-language thesis,
    the English title of a Finnish-language thesis\\
    {\string\studyline}
    & Study line (optional if using the bachelor option)\\
    {\string\tiivistelma}
    & Abstract in Finnish \\
    {\string\abstract}
    & Abstract in English \\
    {\string\avainsanat}
    & Keywords in Finnish \\
    {\string\keywords}
    & Keywords in English \\
    {\string\author}
    & Author's name (if multiple authors, give each their own command 
      -- the {\string\and} command is not supported) \\
    {\string\contactinformation}
    & The contact information of the author \\
    {\string\supervisor}
    & The supervisor of the thesis (if multiple supervisor, give each their own command; optional if using the bachelor option)\\
    \bottomrule
  \end{tabular}
  \caption{Commands for declaring metadata}\label{tbl:metatiedot}
\end{table}
\item If you want, you can write a preface after the
  \string\maketitle\ command.  Use the \string\preface\ to start it.
\item After the preface, if any, you may write a list of terms by
  using the thetermlist environment.  Inside it, you can use the
  \string\item[\textit{term}] command to indicate which term you are
  defining.
\item After \string\maketitle, preface (if any), and term list (if any),
  use the \string\mainmatter\ command.  It will automatically generate
  the tables of contents, figures, and tables that are needful.
\item The commands \string\subsubsection, \string\paragraph{} ja
  \string\subparagraph{} are not supported.
\item Appendices are not \string\chapter s, they are %
  \string\section s.
\item The peceding chapter discussed how to cite sources and geneate a
  bibliography.
\end{itemize}

\chapter{Conclusion}

The last chapter of a thesis is the Conclusion (some authors use
Conculsions, instead).  Keep it short, and discuss what one can
conclude about the thesis statement or research question given in the
Introduction, in light of all that has been written in the thesis.
The Conclusion is also the place to discuss any limitations and
weaknesses of the thesis (especially those that cast doubt on the
reliabliity of the results given in the thesis), if they have not been
already discussed, for example in a Discussion chapter.  It is also
customary to state, what further research might be beneficial in light
of this thesis.

If the Conclusion threatens to become too long, it is a good idea to
split the interpretation of the results into its own chapter, often
called Discussion, making Conclusion short and sweet.

After Conclusion, there is the bibliography, indicated by the
\string\printbibliography\ command, followed by appendices, if any.

\printbibliography

\appendix
\section{Moving from gradu2 to gradu3}

Moving an incomplete thesis from gradu2 to gradu4 is not particularly
difficult.  The first thing to do is to change gradu2 into gradu3 in
the \string\documentclass\ command.  Most of the options given to it
must be removed, as they are not supported.  A ``kandi'' option is
changed into ``bachelor''; any ``english'' option is retained, and so
is ``utf8'', ``latin1'', or ``latin9''.


Table~~\ref{tbl:cmdchange} lists the command name changes that are needed.
A dash indicates that there is no corresponding command.  Note especially the new commands.

\begin{table}[h]\centering
  \begin{tabular}{ll}
    \toprule
    gradu2                 & gradu3  \\
    \midrule
    ---                    & \string\maketitle \\
    ---                    & \string\supervisor \\
    \string\acmccs         & --- \\
    \string\aine           & \string\subject\\
    \string\copyrightowner & --- \\
    \string\fulltitle      & --- \\
    \string\laitos         & \string\department\\
    \string\license        & --- \\
    \string\linja          & \string\studyline\\
    \string\paikka         & --- \\
    \string\setauthor      & \string\author\\
    \string\termlist       & thetermlist environment\\
    \string\tyyppi         & \string\type\\
    \string\yhteystiedot   & \string\contactinformation\\
    \string\yliopisto      & \string\university\\
    \string\ysa            & --- \\
    \bottomrule
  \end{tabular}
  \caption{Command changes from gradu2 to gradu3}
  \label{tbl:cmdchange}
\end{table}

The most effort is likely needed to converting citations and the
bibliography.

\section{Rarely needed features}

In addition to features already mentioned, gradu3 offers the following
additional features:
\begin{itemize}
\item The standard options ``draft'' and ``final''  work.
\item The option ``finnish'' works (but is not needed, as it is the default).
\item You can change the University of the thesis by using the
  \string\university\ command.
\item You can change the Department of the thesis by using the
  \string\department\ command.
\item You can change the formal subject of the thesis by using the
  \string\subject\ command.  In English theses, the subject should be
  prefixed by ``in'' (for example, ``in Information Technology''); in
  Finnish theses, use a capital initial letter and the genitive form
  (``Tietotekniikan'').
\item You can change the type of the thesis by using the \string\type\
  command.
\item You can set the date of the thesis by using the \string\setdate\
  command.  Give it three parameters (day of month, month, and year)
  in numerical form.
\item The chapterquote environment can be used to give an epigraph to
  a chapter.  There is one mandatory parameter (the attribution of the
  epigraph).
\item The command \string\graduclsdate\ prints the release date of the
  current version of gradu3, and the command \string\graduclsversion\
  prints its version number.
\end{itemize}

\end{document}
